%arara: lualatex: { branch: developer, interaction: errorstopmode,
%arara: --> shell: yes, synctex: yes }

% \DocumentMetadata{testphase=phase-III}
\DocumentMetadata{lang=es-MX}

% NOTA: [<+->] se usa comio overlay para hacer que los bullets aparezcan uno por uno.


\documentclass[%
spanish,
progressbar=head,
background=dark,
]{beamer}

\usepackage{babel}
\usepackage{microtype}
\usepackage{tabularray}
\UseTblrLibrary{booktabs}
\usepackage{siunitx}

\usetheme{metropolis}
% \usecolortheme{orchid}
% \title{Técnicas de preparación de materiales}
\title{Aleaciones mecánicas}
\subtitle{Variables del proceso de molienda}
\date{\today}
\author{Pablo E. Alanis}
\institute{Universidad Autónoma de Nuevo Leon, División de Posgrado\\Técnicas de preparación de materiales}
\begin{document}
\maketitle

\section{Variables del proceso}
\begin{frame}{Aleaciones mecánicas | Variables del proceso}
    \begin{itemize}
        \item El proceso de \emph{aleación mecanica} es complejo;
        \item para obtener el producto deseado, se tienen que \textit{optimizar} las condiciones de reacción.
    \end{itemize}
\end{frame}

\begin{frame}{Aleaciones mecánicas | Variables del proceso}
    Entre algunas de las variables que afectan la fase del producto final obtenido, se encuentran:

    \begin{itemize}
        \item \emph{tipo} de molino;
        \item \emph{contenedor} del molino;
        \item \emph{velocidad} de molienda;
        \item \emph{tiempo} de molienda;
        \item \emph{tipo, tamaño y distribución} del medio de molienda;
        \item \emph{relación} en masa de bolas-polvo;
        \item \emph{que tan lleno} está el vial;
        \item \emph{atmósfera} de molienda;
        \item \emph{agente de control} del proceso;
        \item \emph{temperatura} de molienda.
    \end{itemize}
\end{frame}

\begin{frame}{Aleaciones mecánicas | Variables del proceso}
\begin{itemize}
    \item Estas variables no son necesariamente independientes;\\
    \item[] \textbf{por ejemplo:} el tiempo de molienda optimo puede depender de: 
    \begin{enumerate}
        \item tipo de molino;
        \item tamaño del medio de molienda;
        \item temperatura de molienda;
        \item relación bolas-polvo, etc.
    \end{enumerate}
\end{itemize}
\end{frame}

\begin{frame}{Aleaciones mecánicas | Variables del proceso}
    \begin{itemize}
        \item Estas variables no son necesariamente independientes;\\
        \item[] \textbf{por ejemplo:} el tiempo de molienda optimo puede depender de: 
        \begin{enumerate}
            \item tipo de molino;
            \item tamaño del medio de molienda;
            \item temperatura de molienda;
            \item relación bolas-polvo, etc.
        \end{enumerate}
    \end{itemize}
\end{frame}

\subsection{Tipo de molino}

\begin{frame}{Tipos de molinos}
    \begin{itemize}
        \item Existen varios tipos de molinos que pueden usarse según el propósito;
        \item Éstos varían en:
            \begin{enumerate}
                \item capacidad;
                \item velocidad de operación;
                \item capacidad para controlar la temperatura.
            \end{enumerate}
    \end{itemize}
\end{frame}

\begin{frame}{Capacidades de los molinos}
    Según la cantidad de polvo que se requiera sintetizar, se pueden utilizar diferentes molinos:
    \begin{itemize}
        \item \textbf{Para propositos de \emph{screening}} se puede utilizar un molino tipo \emph{SPEX.}
        \item \textbf{Para producir grandes cantidades de polvo} se puede utilizar un molino tipo Fristsch Pulverisette planetario.
    \end{itemize}
\end{frame}

\begin{frame}{Capacidades de los molinos --- Comparación}
\begin{longtblr}[%
    caption = {\small Comparación de tipos de molinos convencionales en función a cantidades de material que pueden procesar.},
    % entry = {Short Caption},
    label = {tbl:TipoDeMolino}]
    {%
    colspec = {XX}, width = 0.85\linewidth,
    rowhead = 1
    }
    \toprule
    Tipo de molino & Tamaño de muestra \\ \midrule
    Molino mezclador & Hasta dos de \qty{20}{\gram} \\
    Molino planetario & Hasta cuatro de \qty{250}{\gram} \\
    Attritores & \qtyrange{0.5}{100}{\kilo\gram} \\
    Molinos Uni-ball & Hasta cuatro de \qty{2000}{\gram} \\ \bottomrule
\end{longtblr}
\end{frame}

\begin{frame}{asdf}
TEMPS
\end{frame}

%Otra edicion

\end{document}